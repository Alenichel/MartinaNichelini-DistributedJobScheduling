%%%%%%%%%%%%%%%%%%%%%%%%%%%%%%%%%%%%%%%%%
% Lachaise Assignment
% LaTeX Template
% Version 1.0 (26/6/2018)
%
% This template originates from:
% http://www.LaTeXTemplates.com
%
% Authors:
% Marion Lachaise & François Févotte
% Vel (vel@LaTeXTemplates.com)
%
% License:
% CC BY-NC-SA 3.0 (http://creativecommons.org/licenses/by-nc-sa/3.0/)
% 
%%%%%%%%%%%%%%%%%%%%%%%%%%%%%%%%%%%%%%%%%

%----------------------------------------------------------------------------------------
%	PACKAGES AND OTHER DOCUMENT CONFIGURATIONS
%----------------------------------------------------------------------------------------

\documentclass{article}

\input{structure.tex} % Include the file specifying the document structure and custom commands
\usepackage{hyperref}

%----------------------------------------------------------------------------------------
%	ASSIGNMENT INFORMATION
%----------------------------------------------------------------------------------------

\title{Distributed System:\\ Distributed Job Scheduluing project report} % Title of the assignment

\author{Yukiko Amagi\\ \texttt{y.amagi@inabauniversity.jp}} % Author name and email address
\author{Stefano Martina\\ \texttt{stefano.martina@mail.polimi.it}
		\\\\Alessandro Nichelini\\ \texttt{alessandro.nichelini@mail.polimi.it} }

\date{Politecnico di Milano\\\today} % University, school and/or department name(s) and a date

%----------------------------------------------------------------------------------------

\begin{document}

\maketitle % Print the title

%----------------------------------------------------------------------------------------
%	INTRODUCTION
%----------------------------------------------------------------------------------------

\section*{Abstract} % Unnumbered section

This document's aim is to describe the main engineering choice we took during the development of the project.

\section*{Assignment}
Implement an infrastructure to manage jobs submitted to a cluster of Executors. Each client may submit a job to any of the executors receiving a job id as a return value. Through such job id, clients may check (contacting the same  executor  they  submitted  the  job  to)  if  the  job  has  been  executed  and  may  retrieve  back  the  results produced by the job.
Executors  communicate  and  coordinate  among  themselves  in  order  to  share  load  such  that  at  each  time  every Executor  is  running  the  same  number  of  jobs  (or  a  number  as  close  as  possible  to  that).  Assume  links  are reliable but processes (i.e., Executors) may fail (and resume back, re-joining the system immediately after). Choose the strategy you find more appropriate to organize communication and coordination. Use stable storage to cope with failures of Executors.
Implement  the  system  in  Java  (or  any  other  language  you  choose)  only  using  basic  communication  facilities (i.e.,  sockets  and  RMI,  in  case  of  Java).  Alternatively,  implement  the  system  in  OMNeT++,  using  an appropriate, abstract model for the system (including the jobs themselves).

\section*{Overview}
\begin{info}
	The project source code is accessible at the following link: \href{https://github.com/Alenichel/MartinaNichelini-DistributedJobScheduling}{https://github.com/Alenichel/MartinaNichelini-DistributedJobScheduling} and it's entirely written in Java 8 using only basic communication facilities.

\end{info}

\paragraph{System description}
The main behavioural aspects are listed below:
\begin{itemize}
	
	\item The system is composed by clients and executors. They respectively refer to the \href{https://github.com/Alenichel/MartinaNichelini-DistributedJobScheduling/blob/master/src/main/java/Main/ClientMain.java}{ClientMain.java} and the \href{https://github.com/Alenichel/MartinaNichelini-DistributedJobScheduling/blob/master/src/main/java/Main/ExecutorMain.java}{ExecutorMain.java} implementation files. There is no limit to the number of them.
	\item There is not a hierarchy between executors, they join the executors "network" by auto-discovering all other running executor
	\item All executors can be contacted by clients. A client needs to know the address of a executor to be able to connect.
	\item The system uses both UDP and TCP sockets for communications between executors and it exploits broadcast UDP packets for auto-discovering. Thus the system is designed to run in local networks inside a single broadcast domain.
	\item Communication between the client and executor uses RMI and TCP sockets.
	\item A client can ask the executor to run a job. It is given with a job id for results retrieval.	
	\item The client can submit jobs that implements the \href{https://github.com/Alenichel/MartinaNichelini-DistributedJobScheduling/blob/master/src/main/java/Tasks/Task.java}{Task/Task.java} interface.

\begin{file}[Task/Task.java]
\begin{lstlisting}[language=Java]
package Tasks;

public interface Task<T> {
    T execute();
}
\end{lstlisting}
\end{file}
	\item Each executor keeps a dictionary of type executorId to \#ofJobs. To keep consistent this data structures among them, each executor notifies all others when its state changes (basically when a job is accepted and completed)
	\item New job requests are collected by the executor to which the client is connected. Requests are forwarded to the executor with the shortest job queue. Executors are multithreading, by default they run two jobs at a time.
	\item The client can ask for job result at any time and to any executor: if the job has been completed, the result will be provided, otherwise the client will be notified about the actual job status.
	\item Job results are kept only by the actual executors. All others have a reference to it. If an executor is asked for a result that it does not have, it will handle the research on behalf of the client.
	\item At startup time, executors check for uncompleted jobs due to previous errors and execute them before joining the system. Executors will also lazily load jobs' result when requested.

		
\end{itemize}
\end{document}
